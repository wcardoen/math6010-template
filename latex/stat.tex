\section{Statistical Analysis}

\subsection{Model}
In what follows we will use the following linear model:
\begin{eqnarray}
       Y_i & = & \beta_0 \, + \, x_{i,1}\,\beta_1 \,+\,
                 x_{i,2}\,\beta_2 \, + \,x_{i,3}\,\beta_3 \,+\,
                 x_{i,4}\,\beta_4 \, + \, \epsilon_i \nonumber \\
	   & = & \displaystyle \sum_{k=0}^4 x_{i,k}\,\beta_k \, + \, \epsilon_i \label{Eqn:Model1}		 
\end{eqnarray}
where $x_{i,0}:=1$.

Eq.\,(\ref{Eqn:Model1}) can also be rewritten in matrix form\footnote{In what follows we will display vectors in bold.}:
\begin{eqnarray}
        \boldsymbol{Y} & = & \boldsymbol{X} \, \boldsymbol{\beta} \, + \, \boldsymbol{\mathcal{E}} \label{Eqn:Model2}
\end{eqnarray}


The test of the null hypothesis can be achieved by calculating the value for the following F-statistic\,\cite{seber2012linear}:
\begin{eqnarray}
        f_{1,n-p} & = & \frac{ \boldsymbol{(A\widehat{\beta}\,-\,c)^T}  \,
                        \boldsymbol{\Bigg[ A(X^TX)^{-1}A^T \Bigg]^{-1} } \,
                          \boldsymbol{(A\widehat{\beta}\,-\,c)}
                         }{S^2} \label{Eq:Fstatistic}
\end{eqnarray}
where the expression $\boldsymbol{(A\widehat{\beta}\,-\,c)}$
imposes a constraint on $\boldsymbol{\widehat{\beta}}$.
